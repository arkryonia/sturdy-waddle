\documentclass[../osfaco.tex]{subfiles}

\title{\large Système informatisé de suivi et d’évaluation des indicateurs de performance de gestion de la réserve de biosphère transfrontalière du Mono}
\author{
	\small Aladji K.M.K\up{1}, Adjonou K\up{1}., Idohou R.\up{2}, Valako K. V.\up{2}, Glele-Kakaï R.\up{2} et Kokou K.\up{1}\\%[1cm]
	\tiny\textit{(1) Faculté des Sciences, Université de Lomé ; 01 BP : 1515 Lome-Togo}\\
	\tiny\textit{(2) Faculté des Sciences Agronomiques, Université d’Abomey-Calavi, 03 B.P. 2819, Cotonou, Bénin}
}\date{}

\begin{document}
	\maketitle
	\textbf{Résumé}
	
	La réserve de biosphère transfrontalière du Mono vise le renforcement de la protection, la conservation et l’utilisation durables des ressources biologiques et des services écosystémiques. Elle contribue au développement durable des communautés du Bénin et du Togo et à l’atténuation des effets adverses du changement climatique. A cet effet, il est préconisé un modèle d’exploitation pour la conservation de la diversité naturelle et culturelle, d’aménagement du territoire, lieu d’expérimentation pour la recherche, la surveillance continue, l’éducation et la formation, etc. Dans l’optique d’optimiser ces services écosystémiques, la présente étude se propose de mettre en place un système informatique de Gestion de Base de Données Relationnelles pour le suivi et l’évaluation de la performance des activités dans la réserve. Ce système intègre 45 indicateurs relatifs aux catégories de données qui sont produites (\textit{tables, tables de jointure, objets géométriques et réseaux d'objets géométriques}) et contient toutes les relations (relations sémantiques, relations de topologie) permettant de les intégrer et les gérer dans un SIG. Il s’appuie sur des applications web et mobile pour faciliter la collecte et la manipulation des données. Les applications communiquent avec la base de données au travers d’un programme d’interface d’application API (Application Programming Intarface). Sur la base de la structuration des données suivant des attributs bien définis, l'interrogation croisée des informations par voie de requêtes SQL, ou encore la localisation précise des objets, des analyses et des requêtes sont réalisables pour évaluer les performances des activités dans la réserve.
	
	\textbf{Most clés}: services écosystémiques, indicateurs de performance, base de données, requêtes SQL, Bénin, Togo
\end{document}